\documentclass[10pt, a4paper]{article}

%\usepackage[margin=2.5cm,nohead]{geometry}
%\renewcommand{\baselinestretch}{1.2} % mudar o espaçamento entre linhas
\usepackage[portuguese]{babel}
\usepackage{indentfirst}
\usepackage[utopia]{mathdesign}     % fonte Utopia para o documento inteiro
\usepackage{inconsolata}            % fonte mono-espaçada
\usepackage[stretch=10]{microtype}  % ajuste de espaçamento entre caracteres
\usepackage{hyperref}               % referências clicáveis
\usepackage{float}                  % posicionamento de figuras
\usepackage{lastpage}               % colocar última página no rodapé
%\usepackage{enumitem}               % modificar listagens
\usepackage{amsmath}                % escrita matemática
\usepackage[extreme]{savetrees}     % aproveitar espaço em branco


\usepackage{multicol}               % múltiplas colunas
\setlength{\columnsep}{1cm}

\usepackage{xcolor}     % mudar a cor do texto
\definecolor{uminho}{RGB}{150, 33, 24}
\definecolor{um_eng}{RGB}{173, 66, 15}

\usepackage{graphicx}     % inserir imagens
\graphicspath{{images/}}  % pasta das imagens

\usepackage{sectsty}          % personalizar capítulos/secções
\chapterfont{\color{uminho}}  % mudar cor dos capítulos
\sectionfont{\color{uminho}}  % mudar cor das secções
 
\usepackage{listings}
\lstset{basicstyle=\footnotesize\ttfamily,breaklines=true,keywordstyle=\color{um_eng}}

%\usepackage{todonotes}              % lista de to-dos
%\setlength{\marginparwidth}{2.5cm}  % tamanho das margens (para to-dos)

\setlength{\parskip}{0.2em}         % espaçamento entre parágrafos



\begin{document} 
\begin{multicols}{2}
[
    \begin{center}
        \textbf{\Large Formulário MDIO 2020/2021}
    \end{center}
]

\thispagestyle{empty}



\section{\texttt{Relax4}}

\subsection{Formato do input}

\begin{lstlisting}
n
m
org dst custo cap (m vezes)
vert (n vezes)
\end{lstlisting}

\begin{itemize}
    \item \textbf{\texttt{n}}: número de vértices
    \item \textbf{\texttt{m}}: número de arcos do grafo
    \item \textbf{\texttt{org}}: vértice de origem do arco
    \item \textbf{\texttt{dst}}: vértice de destino
    \item \textbf{\texttt{custo}}: custo de transporte
    \item \textbf{\texttt{cap}}: capacidade do arco
    \item \textbf{\texttt{vert}}: oferta/procura no vértice, positivo e negativo respetivamente
\end{itemize}



\section{Transportes: Introdução}

\subsection{Modelo geral}

\begin{figure}[H]
    \centering
    \includegraphics[width=0.45\textwidth]{modelo_geral_transportes.png}
\end{figure}

\subsection{Caracterização das soluções básicas}

A uma base podemos associar uma árvore (grafo com vértices não orientados) que suporta todos os vértices.

\subsubsection{Propriedades da árvore de suporte de um grafo G = (V, A)}

\begin{itemize}
    \item é um grafo ligado (existe um caminho entre cada par de vértices)
    \item sem ciclos
    \item com \(|A| = |V| - 1\) (número de arcos = número de vértices - 1)
\end{itemize}

\subsection{Método dos multiplicadores}

\begin{enumerate}
    \item Fixar o valor de qualquer multiplicador em 0
    \item Arcos básicos: \(c_{ij} = u_i - u_j\)
    \item Arcos não-básicos: $\delta_{ij} = c_{ij} - (u_i - u_j)$
\end{enumerate}

\subsection{Pivô}

\begin{figure}[H]
    \centering
    \includegraphics[width=0.35\textwidth]{pivo_intr.png}
\end{figure}



\section{Transportes: Grafos Bipartidos}

Um grafo \(G = (V, A)\) é bipartido se o conjunto de vértices \(V\) puder ser dividido em dois conjuntos disjuntos, \(V_1\) e \(V_2\) (i.e., \(V1 \cup V2 = V,V1 \cap V2 = \emptyset\)), de tal modo que todos os arcos \((i,j) \in A\) tenham origem num vértice \(i \in V_1\) e destino num vértice \(j \in V_2\).

\subsection{Representações}

\begin{figure}[H]
    \centering
    \includegraphics[width=0.45\textwidth]{bipartidos_repr.png}
\end{figure}

\subsection{Solução inicial}

\subsubsection{Método do canto NW}

\begin{enumerate}
    \item Colocar a maior quantidade possível na casa mais a NW \(\implies\)
    \begin{itemize}
        \item ou a procura de um destino (coluna) é totalmente satisfeita,
        \item ou a oferta de uma origem (linha) é totalmente usada,
        \item ou ambas.
    \end{itemize}
    \item Cortar a linha ou a coluna (ou ambas)
    \item Repetir se ainda houver uma casa
\end{enumerate}

\subsubsection{Método do canto NW}

\begin{enumerate}
    \item Colocar a maior quantidade possível na casa com custo mínimo \(\implies\)
    \begin{itemize}
        \item ou a procura de um destino (coluna) é totalmente satisfeita,
        \item ou a oferta de uma origem (linha) é totalmente usada,
        \item ou ambas.
    \end{itemize}
    \item Cortar a linha ou a coluna (ou ambas)
    \item Repetir se ainda houver uma casa
\end{enumerate}

\subsubsection{Seleção da variável básica com valor 0 (quando faltar uma var. básica)}

\begin{figure}[H]
    \centering
    \includegraphics[width=0.45\textwidth]{var_basica_0.png}
\end{figure}

\subsection{Pivô}

\begin{figure}[H]
    \centering
    \includegraphics[width=0.45\textwidth]{bipartidos_pivo.png}
\end{figure}



\section{Transportes: Redes sem capacidades}

\dots (nada?)

\section{Transportes: Redes com capacidades}

\subsection{Caracterização das soluções básicas}

Iguais às referidas no 2.2, mas agora as variáveis no limite superior são também consideradas como não-básicas, para além das iguais a 0.

Uma variável não-básica é atrativa quando:
\begin{itemize}
    \item \(x_{ij} = 0\) (variável aumenta de valor) e \(\delta_{ij} < 0\)
    \item \(x_{ij} = u_{ij}\) (variável decrementa de valor) e \(\delta_{ij} > 0\)
\end{itemize}

\subsection{Transformações}

\subsubsection{Capacidade num vértice}

\begin{figure}[H]
    \centering
    \includegraphics[width=0.45\textwidth]{cap_vertice.png}
\end{figure}

\subsubsection{Limite inferior num arco}

\begin{figure}[H]
    \centering
    \includegraphics[width=0.45\textwidth]{limites.png}
\end{figure}


\section{Programação Inteira: Modelos}

\subsection{Expressões lógicas}

\begin{figure}[H]
    \centering
    \includegraphics[width=0.45\textwidth]{expr_logicas.png}
\end{figure}


\section{Programação Inteira: Planos de corte}

\dots

\section{Programação Inteira: Partição e avaliação}



\end{multicols}
\end{document}